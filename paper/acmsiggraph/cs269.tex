%%% cs269.tex
%%%
%%% This LaTeX source document can be used as the basis for your technical
%%% paper or abstract. Intentionally stripped of annotation, the parameters
%%% and commands should be adjusted for your particular paper - title, 
%%% author, article DOI, etc.
%%% The accompanying ``template.annotated.tex'' provides copious annotation
%%% for the commands and parameters found in the source document. (The code
%%% is identical in ``template.tex'' and ``template.annotated.tex.'')

\documentclass[conference]{acmsiggraph}

\TOGonlineid{45678}
\TOGvolume{0}
\TOGnumber{0}
\TOGarticleDOI{1111111.2222222}
\TOGprojectURL{}
\TOGvideoURL{}
\TOGdataURL{}
\TOGcodeURL{}

\title{Methods in Segmented Image Seam Carving}

\author{
  Weiss, Tomer\\
  \texttt{tweiss@cs.ucla.edu}
  \and
  Liu, Brian\\
  \texttt{liubrian7@ucla.edu}
}



\pdfauthor{Robert A. Smith}

\keywords{Image resizing, Image retargeting, Image seams, Content-aware image manipulation, Display devices}

\begin{document}

%% \teaser{
%%   \includegraphics[height=1.5in]{images/sampleteaser}
%%   \caption{Spring Training 2009, Peoria, AZ.}
%% }

\maketitle

\begin{abstract}

\paragraph{}
Seam carving deals with the task of taking an input picture and resizing it to fit another screen type and aspect ratio. In this paper, we discuss possible methods to improve the algorithm, and also make it work faster. We approached this problem using the segmentation approach. Experimental results demonstrate in some cases our method resizes the image faster and with creating the same or less amount of artifacts as the original algorithm.


 % %Citations can be done this way~\cite{Liu2007} or this more % % %concise 
 % %way~\shortcite{Liu2007}, depending upon the application.

%%Ut wisi enim ad minim veniam, quis nostrud exerci tation ullamcorper
%%suscipit lobortis nisl ut aliquip ex ea commodo consequat. Duis autem
%%vel eum iriure dolor in hendrerit~\cite{Pellacini:2005:LAH}
%%in vulputate velit esse molestie~\cite{notes2002} 
%%consequat, vel illum dolore eu feugiat nulla facilisis at vero eros et
%%accumsan et iusto odio dignissim qui blandit praesent luptatum zzril
%%delenit augue duis dolore te feugait nulla facilisi.~\cite{Park:2006:DSI}

\end{abstract}

\begin{CRcatlist}  

  \CRcat{I.3.0}{Computing Methodologies}{Computer Graphics—General}  
  
  \CRcat{I.4.10}{Computing Methodologies}{Image Processing And Computer Vision}{Image Representation};     
  
\end{CRcatlist}accuracy

\keywordlist

%% Use this only if you're preparing a technical paper to be published in the 
%% ACM 'Transactions on Graphics' journal.

\TOGlinkslist

%% Required for all content. 

\copyrightspace

\section{Introduction}

\paragraph{}
With the recent advances in imaging technology, digital images have become an important component of media distribution. Images are frequently used in news stories, and people post their pictures online to be seen by family and friends. Images, however, are typically authored once, but need to be adapted for consumption under varied conditions. For example, pictures are often displayed on different screens, where the area available for the picture may have a different aspect ratio than the original image has for layout reasons. Dynamically changing the layout of web pages in browsers should take into account the distribution of text and images, resizing them if necessary. Also, the use of thumbnails that faithfully represent the image content is important in image browsing applications. A variety of displays are used for image viewing, ranging from high-resolution computer monitors to TV screens and low-resolution mobile devices. Recently, there has been a growing interest in media retargeting that is driven largely by the growing number of mobile devices used to view digital content. Even if technological advances allow for their resolution to increase, their physical area will still be small. Hence, rearranging the relative sizes of different objects in the image could still provide an improved viewing experience, despite the availability of more pixels.

\paragraph{}
This diversity of image consumption conditions introduces a new problem: images must be resized for optimal display or use in different applications. The process, also known as image re-targeting or image resizing, consists of modifying the image's aspect ratio and size in order to best satisfy the new requirements.  The common approach for all media resizing works is first to define an importance map on the pixels of the media, and then use this map to guide some operator that reduces (or enlarges) the media size. However, straightforward image resizing operators, such as scaling, often do not produce satisfactory results, since they are oblivious to image content. To overcome this limitation, a class of techniques attempt to resize the images in a content-aware fashion, i.e., taking the image content into consideration to preserve important regions and minimize distortions. This is a challenging problem, as it requires preserving the relevant information while maintaining an aesthetically pleasing image for the user.

\begin{figure}[ht]
  \centering
  \includegraphics[width=3.2in]{images/retargeting}
  \caption{Common image retargeting steps.}
\end{figure}

\paragraph{}
There are numerous ways to define importance in media. ~\cite{Liu2010} proposed an approach for changing the composition of objects in a given image in order to improve its aesthetic value, based on rules of thumb from photography such as the rule of thirds. Furthermore, other solutions have been contributed by the computer vision, computer graphics, and human-computer interaction
communities. The definition of \emph{important} can depend on the specific application being considered. There are different approaches for defining importance measures that specify the level of importance of pixels in the image. Also, the definition of what is important and what is unimportant is clearly subjective -- there are situations where user interaction is unavoidable, and many techniques support the specification of important areas as an input provided by the user.


\paragraph{}
One of the automatic image retargeting techniques is Seam Carving, where the general idea is to decrease the image width (or height) one pixel at a time, by removing a seam of minimal importance. Hence Seam Carving can be viewed as a generalized cropping method. A seam is defined as an 8-connected path of pixels (from top to bottom, or from left to right of the image, depending on which dimension is being reduced) that contains only one pixel per row (or column). When the importance map is based on gradient energy, the first removed seam will be in a homogeneous area. The image is then readjusted by shifting pixels left or up to compensate for the removed seam, resulting in an image which is one pixel smaller, either on width or height. So the image changes only at the seam region, while the other areas remain intact. 

\paragraph{}
To summarize, our goal is to experiment with different combinations of Seam Carving and image segmentation, and by so to improve the results of the original paper.


\section{Prior and Related Work}

\paragraph{}
The seam carving technique by ~\cite{Avidan2007} is a popular, recently developed approach for content-aware image resizing. The general idea is to decrease the image width (or height) one pixel at a time,  by removing a seam of minimal importance. Intuitively, if the importance map is based on gradient energy, the first removed seam will be in a homogeneous area. The image is then readjusted by shifting pixels left or up to compensate for the removed seam, resulting in an image which is one pixel smaller, either on width or height. The image changes only at the seam region, while the other areas remain intact. ~\cite{Avidan2007} observe that using gradient energy as the importance map gives satisfactory results, but other importance measures could be used, such as saliency map, entropy, and histograms of oriented gradients. The optimal seams are computed using dynamic programming, and an algorithm for resizing in both dimensions by choosing between optimal vertical or horizontal seams is also presented. The technique can also be used for enlarging the image, by finding seams to be removed and duplicating them. 

\paragraph{}
Generally, the best outcome is achieved when there are enough low-importance seams to be removed, since it creates distortions and artifacts when seams cut through important areas. Moreover, since the energy function reflects feature strength only along to the axes  of image coordinate, it cannot protect some prominent shape boundaries of arbitrary orientations. Also, seam carving does not work well when the input image is feature-rich. To illustrate the limitations of the existing seam carving framework ~\cite{Avidan2007}, we present Figure \ref{fig:orgSeamCarving}, which is obviously feature rich -- the two basketball players have a lot of distinct feature and take large space in the total image's size. Based on the method of ~\cite{Avidan2007}, we attempt to resize this image both in the horizontal and vertical direction.


\begin{figure}[ht]       
    \fbox{\includegraphics[width=1.5in]{images/bird_magic}}   
    \fbox{\includegraphics[width=1.5in]{images/bird_magic_226x375}}
    \caption{Left -- original image. Right -- resized image, result of original Seam Carving method (enlarged). }
    \label{fig:orgSeamCarving}
\end{figure}


 As you could see in Figure \ref{fig:orgSeamCarving}, there are artifacts from result from pixels removed from both players hands and legs. This example shows us that there is much room for improving the original method.

\paragraph{}
In  ~\cite{Avidan2007}, the optimal seams are computed using dynamic programming, and an algorithm for resizing in both dimensions by choosing between optimal vertical or horizontal seams is also presented. The technique can be used for enlarging the image, by finding seams to be removed and duplicating them. It produces impressive results when there are enough low-importance seams to be removed, but creates distortions and artifacts when seams cut through important areas.

\paragraph{}
Motivated by the compelling applications and the challenges related to the problem, we proposed to use image segmentation to improve the seam carving algorithm. There are other methods that rely on segmentation to assign saliencies to different regions in the image. Previous work by ~\cite{Liu2007} and ~\cite{Hasan2009} suggests to segment the image into regions and then assign saliencies to each region by considering heuristics such as the region size, position in the image, and relationships between neighboring regions. ~\cite{Setlur2005} proposed to segment the image using mean-shift and assign saliencies to the obtained regions by a combination of bottom-up and top-down features. Also, ~\cite{Avidan2007} ,authors of the original Seam Carving algorithm, suggested that users could scribble on salient areas. 

\section{Technical details}

\subsection{Overview}

\paragraph{}
Our method builds upon the basis of the original seam carving algorithm, where we the laplacian to find the energy levels of each pixel in the input image. After finding the energy levels, we segment the image to smaller, mini pictures in a grid like fashion -- vertical or horizontal. For example, if we segment the image to 3 by 3 segments, then the we get a total of 9 segments. On each such segment we engage the next step of the seam carving algorithm -- i.e. finding the seams. This step is done via the basic dynamic algorithm for finding seams, but on each segment separately.

\subsection{Image Energy}
Our initial approach, similar to the original paper, is to preserve the image's energy, by removing unnoticeable pixels. That leads to the following energy function that was used in the original paper. We use $I(x)$ to denote the input image, where $x = (x, y)$. 

\begin{equation}
e_I(I)=   \lvert \frac{\partial}{\partial x} I \rvert + \lvert  \frac{\partial}{\partial y} I 
\rvert 
\end{equation}

We need a resizing operator that will be less restrictive than cropping or column removal, but can preserve the image content better than single pixel removals. This leads to our strategy of seam carving and the definition of image seams. Let $I$ be and $n \times m $ image and define a vertical seam to be:

\begin{equation}
s^{x} = { \left\{ s_{i}^{x} \right\}  }_{i=1}^{n} = 
{\left\{ (x(i),i) \right\}}_{i=1}^{n} ,s.t. \forall i \lvert x(i) - x(i - 1)  \rvert \leq 1  
\end{equation}

where x is a mapping $x : [1, . . . , n] \longrightarrow [1, . . . , m]$. That is, a vertical seam is an 8-connected path of pixels in the image from top to bot-tom, containing one, and only one, pixel in each row of the image. Similarly, if y is a mapping y : [1, . . . , m] → [1, . . . , n], then a horizontal seam is:

\begin{equation}
s^{y} = { \left\{ s_{j}^{y} \right\}  }_{j=1}^{m} = 
{\left\{ (y(j),j) \right\} }_{j=1}^{m} ,s.t. \forall j \lvert y(j) - y(j - 1)  \rvert \leq 1
\end{equation} 


The pixels of the path of seam $s$ (e.g. vertical seam $\left\{ s_{i} \right\}$ ) will therefore be $I_s = {\left\{ (I(s_i) \right\} }_{i=1}^{n} = {\left\{ I (x(i),i)) \right\}}_{i=1}^{n} $  . Note that similar to the removal of a row or column from an image, removing the pixels of a seam from an image has only a local effect: all the pixels of the image are shifted left (or up) to compensate for the missing path.


\subsection{Dynamic Programming Algorithm}
Similar to the original method, we find the optimal seam for each segment using dynamic programming. The first step is to traverse the image from the second row to the last row and compute the cumulative minimum energy $M$ for all possible connected seams for each entry $(i, j)$, and $S$ specifies the seam.


\begin{figure}[ht]
  \centering
  \includegraphics[width=2.1in]{images/seam}
  \caption{ Seam example.}
  \label{fig:seamExample}
\end{figure}

\begin{figure}[ht]
  \centering
  \includegraphics[width=2.1in]{images/seam_alg}
  \caption{ Exampe of finding seams using dynamic programming.}
  \label{fig:seamAlg}
\end{figure}


At the end of this process, the minimum value of the last row in $M$ will indicate the end of the minimal connected vertical seam. Therefore, in the second step we backtrack from this minimum entry on $M$ to find the path of the optimal seam. The definition of $M$ for horizontal seams is similar:

\begin{equation}
M(i,j) = e(i,j) + min( M(i-1,j-1), M(i-1,j), M(i-1,j+1) )
\end{equation}

We repeat the above process for each segment, and then remove the newly found seams for each of them, hence resizing the picture.

\subsection{Algorithm Outline}

We build upon the original method ~\cite{Avidan2007}:
\begin{enumerate}
  \item Calculate energy map ~ $O(m n)$
  \item Find seam of lowest energy ~ $O(m n)$
  \item Remove seam / Add seam ~ $O(m n)$
  \item Repeat 1-3 until image is resized
\end{enumerate}

By using the same algorithm on different segments of the image:

\begin{itemize}
  \item Divide the original image into equally sized images and run the original seam carving algorithm.
  \item From each segmented image remove half the number of seams from the original image. This can be done horizontally, vertically, or both, creating a grid like segmentation.
  \item This can be done horizontally, vertically, or both, creating a grid like segmentation.
\end{itemize}


\subsection{Complexity}

\paragraph{}
Running the Seam Carving algorithm on each segment separately, we achieved an improvement in runtime. First, let us remember that finding $s$ seams takes $o(s m n)$ in the original method, where $m,n$ define the image's size and $s$ is the number of seams.

\paragraph{}
With our segmented approach running on a single thread, the runtime time is equal to the total number of segments multiplied by the running time for each segment. So by segmenting the image we can concurrently reduce the running time of the algorithm by a factor of $s s_m s_n$, where $s_m$ and $s_n$ are the number of segments in a column and the number of segments in a row, respectively. So the total number of segments equals to $s_m s_n$, and total number of seams in each segment is $\frac{s}{s_m s_n} $, where $s$ is the absolute difference between the images old width and new width, or equivalently the total number of seams that we add or remove. Total running time for each segment is $ \frac{s m n}{(s_m*s_n)^2} $.


\paragraph{}
Total running time of each segment serially equals to $ s_m s_n   \frac{ O( s m n ) }{ ( s_m s_n)^2  }$. For example, when we segment the image into 4 parts, we get $s_m,s_n=2$, and $\frac{s}{4}$. So in this case: $\frac{O(s*m*n)}{16}$, and the the total running time for each segment is $4 * \frac{O (s m n)}{16} = \frac{O(s m n)}{4}$, which runs serially. But there is potential for a significant speedup if we run the algorithm concurrently -- up to $\frac{O(s*m*n)}{16}$.



\subsection{Implementation}

\paragraph{}
We initially implemented the original Seam Carving algorithm in Java, but encountered some difficulties when trying to expend the code to support image segmentation. Therefore we rewrote the original and segmentation expansion of in python. 

Access to the code is available through our paper website  \href{https://github.com/tomerwei/seamCarving/blob/master/paper/acmsiggraph/cs269.pdf?raw=true}{website}


\section{Experimental Results}

We took a number of images, some of which were used previous papers, varying from feature rich images to images that show great results with the original seam carving method. All experiments were run on a standard home laptop, equivalent to MacBook Air.


\begin{figure}[ht]       
    \fbox{\includegraphics[width=3.5in]{images/beach_results}}   
    \caption{Left -- Input image. Right -- output of segmented seam carving. Beach Image}
    \label{materialflowChart}
\end{figure}


\begin{figure}[ht]
  \centering
  \includegraphics[width=3.2in]{images/beach}
  \caption{ Run-times}
  \label{fig:runtimes}
\end{figure}


As we see in Figure \ref{fig:runtimes}, segmenting images into equally sized segments results in a much faster running algorithm, even if the algorithm runs serially. We also observed that by increasing number of grid segments we speed up the algorithm, and decrease artifacts within an image, as seen in Figure \ref{fig:bird3232}.


\begin{figure}[ht]
  \centering
  \includegraphics[width=1.8in]{images/bird_32by32}
  \caption{ Segmented Seam Carving(30x30)
  }
  \label{fig:bird3232}
\end{figure}


Finally, we notice that as we segment to smaller and smaller parts, black artifacts appear in the images edges. We hypothesize that this is a byproduct of our implementation, but couldn't confirm it.


\section{Limitations}

The proposed method is dependent on where the image is segmented. So if the segmentation cut through feature rich parts of the image, some details might be lost and artifacts might be created. 

\begin{figure}[ht]       
    \fbox{\includegraphics[width=1.5in]{images/hof}}   
    \fbox{\includegraphics[width=1.5in]{images/hof_segemented}}
    \caption{Left -- input image. Right -- output of segmented seam carving, artifacts created are circled.}
    \label{fig:Beach_in_and_out}
    
\end{figure}

For example, we used 3 by 3 segmentation to produce Figure ~\ref{fig:Beach_in_and_out}. The artifacts are the results of the segmentation lines going through the coast line.


\begin{figure}[ht]       
    \fbox{\includegraphics[width=1.5in]{images/hof_segemented_lines}}
    \caption{Left -- input image. Right -- output of segmented seam carving, artifacts created are circled.}
    \label{fig:Beach_lines}
\end{figure}

Figure ~\ref{fig:Beach_in_and_out} illustrates the segmentation we used. We shall discuss how to pick better segmentation lines in the next section.


\section{Conclusions and Future Work}

\paragraph{}
In this paper we proposed a content-aware im-
age resizing algorithm which builds upon the previous work by ~\cite{Avidan2007}. Our segmentation approach shows that there is much to be done in improving the speed runtime and performance of the original method. We didn't have enough time to investigate other methods to segment a picture, like image saliency as shown in ~\cite{Liu2007} and ~\cite{Hasan2009}. Here are some points that we think are worth investigating in future work:

\begin{itemize}
  \item Allowing segmentation into non-rectangular regions could lead to better results, especially if the segmentation fits the features and image energy map, so the seams that we create will only carve through these regions, thereby reducing artifacts in the output resized image.
  \item Flexible user input for segmentation is also one possible improvement for segmented Seam Carving -- the user will be able to arrange the segments, and also mark areas of importance in the image so that the seam will skip them, thus avoiding  artifacts. 
  \item Parallelizing our python implementation for greater speed increase.
  \item As wee in Figure \ref{fig:bird3232}, some artifacts are created between the different segments. Applying a small smoothing operator to areas in between segments to reduce artifacts could lead to better results.
  \item Using forward instead of backward energy, so that instead of removing seams with lowest energy, we remove seam which leaves the image with highest energy.
  \item Applying the saliency filter before seam carving to emphasize areas to minimize distortion.  
\end{itemize}


\paragraph{}
Finally, there is always room for testing on more images, and trying different energy filters, since the results tend to be subjective by nature. There are further ways to utilize image segmentation to improve image resizing in large and specifically the Seam Carving method, and conducting more experiments and getting input from users is essential in refining this technique. 

\bibliographystyle{acmsiggraph}
\bibliography{template}
\end{document}
